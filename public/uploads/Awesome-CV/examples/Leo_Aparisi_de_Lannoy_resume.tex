%!TEX TS-program = xelatex
%!TEX encoding = UTF-8 Unicode
% Awesome CV LaTeX Template for CV/Resume
%
% This template has been downloaded from:
% https://github.com/posquit0/Awesome-CV
%
% Author:
% Claud D. Park <posquit0.bj@gmail.com>
% http://www.posquit0.com
%
% Template license:
% CC BY-SA 4.0 (https://creativecommons.org/licenses/by-sa/4.0/)
%


%-------------------------------------------------------------------------------
% CONFIGURATIONS
%-------------------------------------------------------------------------------
\documentclass[11pt, letterpaper]{awesome-cv}
\providecolor{url}{HTML}{000000}
\providecolor{link}{HTML}{000000}
\providecolor{cite}{HTML}{000000}
% Define a section for CV
% Usage: \cvsection{<section-title>}
\hypersetup{pdfauthor={Leo Aparisi de Lannoy}
pdftitle={},
pdfkeywords={},
pdfsubject={},
pdfcreator={},
pdflang={English},
breaklinks=true,
colorlinks=true,
linkcolor=link,
urlcolor=url,
citecolor=cite
}
% Configure page margins with geometry
\geometry{left=1cm, top=.8cm, right=1cm, bottom=1.5cm, footskip=.5cm}

% Color for highlights
% Awesome Colors: awesome-emerald, awesome-skyblue, awesome-red, awesome-pink, awesome-orange
%                 awesome-nephritis, awesome-concrete, awesome-darknight
%\colorlet{awesome}{awesome-red}
% Uncomment if you would like to specify your own color
% Colors for text
% Uncomment if you would like to specify your own color
% \definecolor{darktext}{HTML}{414141}
% \definecolor{text}{HTML}{333333}
% \definecolor{graytext}{HTML}{5D5D5D}
% \definecolor{lighttext}{HTML}{999999}
% \definecolor{sectiondivider}{HTML}{5D5D5D}


% Set false if you don't want to highlight section with awesome color
\setbool{acvSectionColorHighlight}{false}

% If you would like to change the social information separator from a pipe (|) to something else
\renewcommand{\acvHeaderSocialSep}{\quad\textbar\quad}


%-------------------------------------------------------------------------------
%	PERSONAL INFORMATION
%	Comment any of the lines below if they are not required
%-------------------------------------------------------------------------------
% Available options: circle|rectangle,edge/noedge,left/right
% \photo{./examples/profile.png}
\name{Léo}{Aparisi de Lannoy}
\mobile{(+1)312-394-9854}
\email{laparisidelannoy@uchicago.edu}
%\dateofbirth{07-27-1995}
\extrainfo{French citizen (F1 visa)}
\homepage{leoadl.com}
\github{leoadl}
\linkedin{leoadl}
 \googlescholar{ZrkvVt0AAAAJ}{Scholar}
% \gitlab{gitlab-id}
% \stackoverflow{SO-id}{SO-name}
% \twitter{@twit}
% \skype{skype-id}
% \reddit{reddit-id}
% \medium{madium-id}
% \kaggle{kaggle-id}
% \googlescholar{googlescholar-id}{name-to-display}
%% \firstname and \lastname will be used
% \extrainfo{extra information}


%-------------------------------------------------------------------------------
\begin{document}

% Print the header with above personal information
% Give optional argument to change alignment(C: center, L: left, R: right)
\makecvheader[L]

% Print the footer with 3 arguments(<left>, <center>, <right>)
% Leave any of these blank if they are not needed
\makecvfooter
{\today}
{Léo Aparisi de Lannoy~~~·~~~Ph.D. candidate}
{\thepage}


%-------------------------------------------------------------------------------
%	CV/RESUME CONTENT
%	Each section is imported separately, open each file in turn to modify content
%-------------------------------------------------------------------------------
%-------------------------------------------------------------------------------
%	SECTION TITLE
%-------------------------------------------------------------------------------
\cvsection{Summary}


%-------------------------------------------------------------------------------
%	CONTENT
%-------------------------------------------------------------------------------
\begin{cvparagraph}
	Ph.D. in Financial Economics, with a specialization in Macroeconomics and Asset Pricing, eager to apply his skills in Statistics, Economics, and Programming to quantitative challenges. Seeking positions in Quant Finance, starting in Summer 2024.
\end{cvparagraph}

%-------------------------------------------------------------------------------
%	SECTION TITLE
%-------------------------------------------------------------------------------
\cvsection{Education}


%-------------------------------------------------------------------------------
%	CONTENT
%-------------------------------------------------------------------------------
\begin{cventries}

	%---------------------------------------------------------
	\cventry
	{Ph.D. in Financial Economics} % Degree
	{University of Chicago} % Institution
	{Chicago, USA} % Location
	{2018 - 2024 (expected)} % Date(s)
	{\begin{cvitems}
			\item Dissertation on \textit{Asset Pricing Implications of Monetary Policy Normalization}. Specialization in \textbf{Macroeconomics} \& \textbf{Asset Pricing}.
		\end{cvitems}}

	\cventry
	{M.Sc. Analysis and Policy in Economics, \textbf{summa cum laude}} % Degree
	{Paris School of Economics} % Institution
	{Paris, France} % Location
	{2016 - 2018} % Date(s)
	{}

	\cventry
	{B.Sc. in Physics, \textbf{cum laude}} % Degree
	{Ecole Normale Superieure Ulm} % Institution
	{Paris, France} % Location
	{2015 - 2016} % Date(s)
	{}

	%---------------------------------------------------------
\end{cventries}

%-------------------------------------------------------------------------------
%	SECTION TITLE
%-------------------------------------------------------------------------------
\cvsection{Experience}


%-------------------------------------------------------------------------------
%	CONTENT
%-------------------------------------------------------------------------------
\begin{cventries}
	\cventry
	{Topics in Economics}
	{Instructor}
	{University of Chicago}
	{2021}
	{
			 Designed and delivered lectures for Master students in Financial Mathematics on macroeconomics, and dynamic asset pricing.
		}


	\cventry
	{Empirical Analysis II; Money, Banking, and the Financial Crisis; Financial Markets in the Macroeconomy; Risk, Uncertainty, and Value; Monetary Economics I; Theory of Income I}
	{Teaching Assistant}
	{University of Chicago}
	{2019 - 2022}
	{
			 Assisted PhD and Executive MBA level classes on macroeconomics, time series econometrics, and dynamic programming.
		}

	\cventry
	{Lars Peter Hansen \& Thomas J. Sargent, Ufuk Akcigit}
	{Research Assistant}
	{University of Chicago}
	{2019 - 2020}
	{
			 Developed a quantitative model of the optimal taxation for R\&D Policies in the US using Numpy and Scipy.
		}
\end{cventries}

%-------------------------------------------------------------------------------
%	SECTION TITLE
%-------------------------------------------------------------------------------
\cvsection{Research Experience}


%-------------------------------------------------------------------------------
%	CONTENT
%-------------------------------------------------------------------------------
\begin{cventries}
	\cventry
	{joint with Anmol Bhandari, David Evans, Mikhail Golosov and Thomas J. Sargent} % Degree
	{\href{https://static1.squarespace.com/static/54c19f18e4b0ef5f4b9f8dae/t/6328ee5213a65c43a48423f8/1663626851912/abegs4draft.pdf}{Managing Public Portfolios} \textit{(Revise \& Resubmit at The Journal of Political Economy)}}  % Institution
	{\href{https://www.nber.org/papers/w30501}{NBER Working Paper \#30501}} % Location
	{2022} % Date(s)
	{\begin{cvitems}
			\item Helped develop a quantitative framework for optimally managing public portfolios for various market structures for financial assets, and trading frictions.
			\item Contributed to characterize numerically the optimal US maturity structure using macro and bonds market data. This calibration shows that the \textit{interest rate risk} should shape the US debt portfolio.
			\item Implemented an affine dynamic asset pricing model of the US government bond market in Python (Pandas, Numpy, Scipy).
		\end{cvitems}}

	\cventry
	{Lars Peter Hansen \& Thomas J. Sargent, Ufuk Akcigit}
	{Research Assistant}
	{University of Chicago}
	{2019-2020}
	{\begin{cvitems}
			\item Helped code a quantitative model of the optimal taxation for R\&D Policies in the US using Numpy and Scipy. Extended simulations of a macroeconomic model of an economy with rich heterogeneity using Julia.
			\item Checked proofs on statistical uncertainty, model misspecification, and time inconsistency, using concepts of statistical divergences, information geometry, and entropy.
		\end{cvitems}
	}
\end{cventries}

%-------------------------------------------------------------------------------
%	SECTION TITLE
%-------------------------------------------------------------------------------
\cvsection{Honors \& Awards }


%-------------------------------------------------------------------------------
%	CONTENT
%-------------------------------------------------------------------------------
\begin{cvhonors}

	%---------------------------------------------------------
	\cvhonor
	{Martin C. And Margaret M. Lee Prize} % Award
	{Best Performance in the Graduate Macroeconomics Sequence} % Event
	{Chicago, USA} % Location
	{2019} % Date(s)
	\cvhonor
	{Neubauer Fellowship} % Award
	{Graduate Fellowship} % Event
	{Chicago, USA} % Location
	{2018} % Date(s)

	\cvhonor
	{First Prize} % Award
	{French National History Competition (Concours General)} % Event
	{Paris, France} % Location
	{2012} % Date(s)

	%---------------------------------------------------------
\end{cvhonors}

%-------------------------------------------------------------------------------
%	SECTION TITLE
%-------------------------------------------------------------------------------
\cvsection{Skills}


%-------------------------------------------------------------------------------
%	CONTENT
%-------------------------------------------------------------------------------
\begin{cvskills}

	\cvskill
	{Programming} % Category
	{Python (Numpy, Scipy, Pandas, Pola-rs, Matplotlib, Seaborn, scikit-learn, PyTorch, JAX), Julia (DataFrames, JuMP, Plots), Bash} % Skills

	\cvskill
	{Software}
	{CLI/Unix, Linux (Debian), Virtualization (Proxmox, LXC), Docker, ZFS, S3 Storage, Git, Wireguard VPN, Vim/Neovim, \LaTeX, Pandoc Markdown}

	\cvskill
	{Data}
	{OLS, ARMA, Machine Learning, Deep Learning, Fourier Analysis, Maximum Likelihood, Generalized Method Moments}
	%---------------------------------------------------------
	\cvskill
	{Languages} % Category
	{French (Native), English (Fluent), Spanish (Proficient)} % Skills

	\cvskill
	{Hobbies}
	{Coffee Barista, Cooking, Soccer, Travelling, Self-Hosting, Reading about History, Physics Videos}
	%---------------------------------------------------------
\end{cvskills}

% \input{resume/presentation.tex}
% \input{resume/writing.tex}
% \input{resume/committees.tex}
% \input{resume/extracurricular.tex}


%-------------------------------------------------------------------------------
\end{document}
