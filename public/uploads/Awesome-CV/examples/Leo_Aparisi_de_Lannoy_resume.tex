%!TEX TS-program = xelatex
%!TEX encoding = UTF-8 Unicode
% Awesome CV LaTeX Template for CV/Resume
%
% This template has been downloaded from:
% https://github.com/posquit0/Awesome-CV
%
% Author:
% Claud D. Park <posquit0.bj@gmail.com>
% http://www.posquit0.com
%
% Template license:
% CC BY-SA 4.0 (https://creativecommons.org/licenses/by-sa/4.0/)
%


%-------------------------------------------------------------------------------
% CONFIGURATIONS
%-------------------------------------------------------------------------------
\documentclass[11pt, letterpaper]{awesome-cv}
\providecolor{url}{HTML}{000000}
\providecolor{link}{HTML}{000000}
\providecolor{cite}{HTML}{000000}
% Define a section for CV
% Usage: \cvsection{<section-title>}
\hypersetup{pdfauthor={Leo Aparisi de Lannoy}
pdftitle={},
pdfkeywords={},
pdfsubject={},
pdfcreator={},
pdflang={English},
breaklinks=true,
colorlinks=true,
linkcolor=link,
urlcolor=url,
citecolor=cite
}
% Configure page margins with geometry
\geometry{left=1cm, top=.8cm, right=1cm, bottom=1.5cm, footskip=.5cm}

% Color for highlights
% Awesome Colors: awesome-emerald, awesome-skyblue, awesome-red, awesome-pink, awesome-orange
%                 awesome-nephritis, awesome-concrete, awesome-darknight
%\colorlet{awesome}{awesome-red}
% Uncomment if you would like to specify your own color
% Colors for text
% Uncomment if you would like to specify your own color
% \definecolor{darktext}{HTML}{414141}
% \definecolor{text}{HTML}{333333}
% \definecolor{graytext}{HTML}{5D5D5D}
% \definecolor{lighttext}{HTML}{999999}
% \definecolor{sectiondivider}{HTML}{5D5D5D}


% Set false if you don't want to highlight section with awesome color
\setbool{acvSectionColorHighlight}{false}

% If you would like to change the social information separator from a pipe (|) to something else
\renewcommand{\acvHeaderSocialSep}{\quad\textbar\quad}


%-------------------------------------------------------------------------------
%	PERSONAL INFORMATION
%	Comment any of the lines below if they are not required
%-------------------------------------------------------------------------------
% Available options: circle|rectangle,edge/noedge,left/right
% \photo{./examples/profile.png}
\name{Léo}{Aparisi de Lannoy}
\mobile{(+1)312-394-9854}
\email{laparisidelannoy@uchicago.edu}
%\dateofbirth{07-27-1995}
\extrainfo{French citizen (F1 visa)}
\homepage{leoadl.com}
\github{leoadl}
\linkedin{leoadl}
 \googlescholar{ZrkvVt0AAAAJ}{Scholar}
% \gitlab{gitlab-id}
% \stackoverflow{SO-id}{SO-name}
% \twitter{@twit}
% \skype{skype-id}
% \reddit{reddit-id}
% \medium{madium-id}
% \kaggle{kaggle-id}
% \googlescholar{googlescholar-id}{name-to-display}
%% \firstname and \lastname will be used
% \extrainfo{extra information}


%-------------------------------------------------------------------------------
\begin{document}

% Print the header with above personal information
% Give optional argument to change alignment(C: center, L: left, R: right)
\makecvheader[L]

% Print the footer with 3 arguments(<left>, <center>, <right>)
% Leave any of these blank if they are not needed
\makecvfooter
{\today}
{Léo Aparisi de Lannoy~~~·~~~Ph.D. candidate}
{\thepage}


%-------------------------------------------------------------------------------
%	CV/RESUME CONTENT
%	Each section is imported separately, open each file in turn to modify content
%-------------------------------------------------------------------------------
%-------------------------------------------------------------------------------
%	SECTION TITLE
%-------------------------------------------------------------------------------
\cvsection{Work Experience}

\begin{cventries}
  \cventry
    {Quantitative Researcher} % Job title
    {Squarepoint Capital} % Organization
    {NYC, NY} % Location
    {July 2024 -} % Date(s)
    {}
\end{cventries}

%-------------------------------------------------------------------------------
%	SECTION TITLE
%-------------------------------------------------------------------------------
\cvsection{Education}


%-------------------------------------------------------------------------------
%	CONTENT
%-------------------------------------------------------------------------------
\begin{cventries}

	%---------------------------------------------------------
	\cventry
	{Ph.D. in Financial Economics} % Degree
	{University of Chicago} % Institution
	{Chicago, USA} % Location
	{September 2018 - June 2024 (expected)} % Date(s)
	{Advisors: \textit{\href{https://alvarezfernando.com/}{Fernando Alvarez}, \href{https://voices.uchicago.edu/golosov/contact/}{Mikhail Golosov}, \href{https://larspeterhansen.org/contact/}{Lars Peter Hansen}, \href{https://voices.uchicago.edu/veronicaguerrieri/contact/}{Veronica Guerrieri}}}

	\cventry
	{M.A. in Economics} % Degree
	{University of Chicago} % Institution
	{Chicago, USA} % Location
	{September 2018 - June 2020} % Date(s)
	{Primary Fields:
		\textbf{Macroeconomics}, \textbf{Monetary Economics}
	}

	\cventry
	{M.Sc. Analysis and Policy in Economics} % Degree
	{Paris School of Economics} % Institution
	{Paris, France} % Location
	{September 2016 - June 2018} % Date(s)
	{Summa cum laude}

	\cventry
	{B.Sc. in Physics} % Degree
	{Ecole Normale Superieure Ulm} % Institution
	{Paris, France} % Location
	{September 2015 - June 2016} % Date(s)
	{Cum laude}

	%---------------------------------------------------------
\end{cventries}

%-------------------------------------------------------------------------------
%	SECTION TITLE
%-------------------------------------------------------------------------------
\cvsection{Experience}


%-------------------------------------------------------------------------------
%	CONTENT
%-------------------------------------------------------------------------------
\begin{cventries}
	\cventry
	{Topics in Economics}
	{Instructor}
	{University of Chicago}
	{2021}
	{\begin{cvitems}
			\item Designed and delivered lectures for Master students in Financial Mathematics on macroeconomics, and dynamic asset pricing.
		\end{cvitems}}


	\cventry
	{Empirical Analysis II; Money, Banking, and the Financial Crisis; Financial Markets in the Macroeconomy; Risk, Uncertainty, and Value; Monetary Economics I; Theory of Income I}
	{Teaching Assistant}
	{University of Chicago}
	{2019 - 2022}
	{\begin{cvitems}
			\item Assisted PhD and Executive MBA level classes on macroeconomics, time series econometrics, and dynamic programming.
		\end{cvitems}}

	\cventry
	{Lars Peter Hansen \& Thomas J. Sargent, Ufuk Akcigit}
	{Research Assistant}
	{University of Chicago}
	{2019 - 2020}
	{\begin{cvitems}
			\item Developed a quantitative model of the optimal taxation for R\&D Policies in the US using Numpy and Scipy.
		\end{cvitems}}
\end{cventries}

%-------------------------------------------------------------------------------
%	SECTION TITLE
%-------------------------------------------------------------------------------
\cvsection{Experience}


%-------------------------------------------------------------------------------
%	CONTENT
%-------------------------------------------------------------------------------
\begin{cventries}

	%---------------------------------------------------------
	\cventry
	{Site Reliability Engineer \& Infrastructure Team Lead} % Job title
	{Danggeun Pay Inc. (KarrotPay)} % Organization
	{Seoul, S.Korea} % Location
	{Mar. 2021 - Present} % Date(s)
	{
		\begin{cvitems} % Description(s) of tasks/responsibilities
			\item {Everything that matters.}
			\item {Designed and provisioned the entire infrastructure on the AWS cloud to meet security compliance and acquire a business license for financial services in Korea.}
			\item {Continuously improved the infrastructure architecture since launching the service. (currently 3.6 million users)}
		\end{cvitems}
	}

	%---------------------------------------------------------
	\cventry
	{Site Reliability Engineer} % Job title
	{Danggeun Market Inc.} % Organization
	{Seoul, S.Korea} % Location
	{Feb. 2021 - Mar. 2021} % Date(s)
	{
	}

	%---------------------------------------------------------
	\cventry
	{Founding Member \& Director of Infrastructure Division} % Job title
	{Kasa} % Organization
	{Seoul, S.Korea} % Location
	{Jun. 2018 - Jan. 2021} % Date(s)
	{
		\begin{cvitems} % Description(s) of tasks/responsibilities
			\item {Designed on-boarding process to guide new engineers, help them to focus on the right tasks, and set expectations to help them be successful at Infrastructure team.}
			\item {Migrated the orchestration system from DC/OS to Kubernetes which is based on AWS EKS. Managed 3 Kubernetes clusters and 300+ pods. Managed all Kubernetes manifests declaratively with Kustomize and ArgoCD.}
			\item {Designed and managed complex network configurations on AWS with 4 VPC and 100+ subnets. Separated the development network and operation network according to financial regulations. Established dedicated network connections from AWS VPC to partners' on-premise network based on AWS Direct Connect with secure connection using IPsec VPN. Provisioned OpenVPN servers with LDAP integration.}
			\item {Provisioned a observability system with Kafka, Elastic Stack(Filebeat, Heartbeat, APM Server, Logstash, Elasticsearch, Kibana). Collected log, uptime, tracing data from hosts, containers, pods and more. The ES cluster which has 9 nodes processed more than 1 billion documents per month. Wrote Terraform module to easily provision ES cluster on AWS EC2 instances.}
			\item {Provisioned a monitoring system with Kafka, Telegraf, InfluxDB, Grafana. Collected metrics from hosts, containers, pods and more. Wrote Terraform module to easily provision InfluxDB with HA on AWS EC2 instances.}
			\item {Introduced Kong API Gateway to easily connect all API microservices with a declarative management method based on Terraform and Atlantis to collaborate and audit change history.}
			\item {Provisioned the Directory Service for employee identity management based on OpenLDAP which guarantees HA with multi-master replication.}
			\item {Implemented Worker microservices consuming Kafka event topics for email, SMS, Kakaotalk and Slack notification. Developed in-house framework to easily build Kafka consumer microservice with common features including retry on failure, DLQ(Dead Letter Queue), event routing and more.}
			\item {Introduced Elastic APM to help distributed tracing, trouble-shooting and performance testing in MSA.}
		\end{cvitems}
	}

	%---------------------------------------------------------
	\cventry
	{Software Architect} % Job title
	{Omnious. Co., Ltd.} % Organization
	{Seoul, S.Korea} % Location
	{Jun. 2017 - May 2018} % Date(s)
	{
		\begin{cvitems} % Description(s) of tasks/responsibilities
			\item {Provisioned an easily managable hybrid infrastructure(Amazon AWS + On-premise) utilizing IaC(Infrastructure as Code) tools like Ansible, Packer and Terraform.}
			\item {Built fully automated CI/CD pipelines on CircleCI for containerized applications using Docker, AWS ECR and Rancher.}
			\item {Designed an overall service architecture and pipelines of the Machine Learning based Fashion Tagging API SaaS product with the micro-services architecture.}
			\item {Implemented several API microservices in Node.js Koa and in the serverless AWS Lambda functions.}
			\item {Deployed a centralized logging environment(ELK, Filebeat, CloudWatch, S3) which gather log data from docker containers and AWS resources.}
			\item {Deployed a centralized monitoring environment(Grafana, InfluxDB, CollectD) which gather system metrics as well as docker run-time metrics.}
		\end{cvitems}
	}

	%---------------------------------------------------------
	\cventry
	{Co-founder \& Software Engineer} % Job title
	{PLAT Corp.} % Organization
	{Seoul, S.Korea} % Location
	{Jan. 2016 - Jun. 2017} % Date(s)
	{
		\begin{cvitems} % Description(s) of tasks/responsibilities
			\item {Implemented RESTful API server for car rental booking application(CARPLAT in Google Play).}
			\item {Built and deployed overall service infrastructure utilizing Docker container, CircleCI, and several AWS stack(Including EC2, ECS, Route 53, S3, CloudFront, RDS, ElastiCache, IAM), focusing on high-availability, fault tolerance, and auto-scaling.}
			\item {Developed an easy-to-use Payment module which connects to major PG(Payment Gateway) companies in Korea.}
		\end{cvitems}
	}

	%---------------------------------------------------------
	\cventry
	{Researcher} % Job title
	{Undergraduate Research, Machine Learning Lab(Prof. Seungjin Choi)} % Organization
	{Pohang, S.Korea} % Location
	{Mar. 2016 - Exp. Jun. 2017} % Date(s)
	{
		\begin{cvitems} % Description(s) of tasks/responsibilities
			\item {Researched classification algorithms(SVM, CNN) to improve accuracy of human exercise recognition with wearable device.}
			\item {Developed two TIZEN applications to collect sample data set and to recognize user exercise on SAMSUNG Gear S.}
		\end{cvitems}
	}

	%---------------------------------------------------------
	\cventry
	{Software Engineer \& Security Researcher (Compulsory Military Service)} % Job title
	{R.O.K Cyber Command, MND} % Organization
	{Seoul, S.Korea} % Location
	{Aug. 2014 - Apr. 2016} % Date(s)
	{
		\begin{cvitems} % Description(s) of tasks/responsibilities
			\item {Lead engineer on agent-less backtracking system that can discover client device's fingerprint(including public and private IP) independently of the Proxy, VPN and NAT.}
			\item {Implemented a distributed web stress test tool with high anonymity.}
			\item {Implemented a military cooperation system which is web based real time messenger in Scala on Lift.}
		\end{cvitems}
	}

	%---------------------------------------------------------
	\cventry
	{Software Engineer} % Job title
	{ShitOne Corp.} % Organization
	{Seoul, S.Korea} % Location
	{Dec. 2011 - Feb. 2012} % Date(s)
	{
		\begin{cvitems} % Description(s) of tasks/responsibilities
			\item {Developed a proxy drive smartphone application which connects proxy driver and customer. Implemented overall Android application logic and wrote API server for community service, along with lead engineer who designed bidding protocol on raw socket and implemented API server for bidding.}
		\end{cvitems}
	}

	%---------------------------------------------------------
\end{cventries}

%-------------------------------------------------------------------------------
%	SECTION TITLE
%-------------------------------------------------------------------------------
\cvsection{Honors \& Awards}


%-------------------------------------------------------------------------------
%	SUBSECTION TITLE
%-------------------------------------------------------------------------------
\cvsubsection{International Awards}


%-------------------------------------------------------------------------------
%	CONTENT
%-------------------------------------------------------------------------------
\begin{cvhonors}

	{2018} % Date(s)

	\cvhonor
	{First Prize} % Award
	{French National History Competition (Concours General)} % Event
	{Paris, France} % Location
	{2012} % Date(s)

	%---------------------------------------------------------
\end{cvhonors}


%-------------------------------------------------------------------------------
%	SUBSECTION TITLE
%-------------------------------------------------------------------------------
\cvsubsection{Domestic Awards}


%-------------------------------------------------------------------------------
%	CONTENT
%-------------------------------------------------------------------------------
\begin{cvhonors}

	%---------------------------------------------------------
	\cvhonor
	{2nd Place} % Award
	{AWS Korea GameDay} % Event
	{Seoul, S.Korea} % Location
	{2021} % Date(s)

	%---------------------------------------------------------
	\cvhonor
	{3rd Place} % Award
	{WITHCON Hacking Competition Final} % Event
	{Seoul, S.Korea} % Location
	{2015} % Date(s)

	%---------------------------------------------------------
	\cvhonor
	{Silver Prize} % Award
	{KISA HDCON Hacking Competition Final} % Event
	{Seoul, S.Korea} % Location
	{2017} % Date(s)

	%---------------------------------------------------------
	\cvhonor
	{Silver Prize} % Award
	{KISA HDCON Hacking Competition Final} % Event
	{Seoul, S.Korea} % Location
	{2013} % Date(s)

	%---------------------------------------------------------
	\cvhonor
	{2nd Award} % Award
	{HUST Hacking Festival} % Event
	{S.Korea} % Location
	{2013} % Date(s)

	%---------------------------------------------------------
	\cvhonor
	{3rd Award} % Award
	{HUST Hacking Festival} % Event
	{S.Korea} % Location
	{2010} % Date(s)

	%---------------------------------------------------------
	\cvhonor
	{3rd Award} % Award
	{Holyshield 3rd Hacking Festival} % Event
	{S.Korea} % Location
	{2012} % Date(s)

	%---------------------------------------------------------
	\cvhonor
	{2nd Award} % Award
	{Holyshield 3rd Hacking Festival} % Event
	{S.Korea} % Location
	{2011} % Date(s)

	%---------------------------------------------------------
	\cvhonor
	{5th Place} % Award
	{PADOCON Hacking Competition Final} % Event
	{Seoul, S.Korea} % Location
	{2011} % Date(s)

	%---------------------------------------------------------
\end{cvhonors}

%-------------------------------------------------------------------------------
%	SUBSECTION TITLE
%-------------------------------------------------------------------------------
\cvsubsection{Community}


%-------------------------------------------------------------------------------
%	CONTENT
%-------------------------------------------------------------------------------
\begin{cvhonors}

	%---------------------------------------------------------
	\cvhonor
	{AWS Community Builder (Container)} % Award
	{Amazon Web Services (AWS)} % Event
	{} % Location
	{2022} % Date(s)

	%---------------------------------------------------------
	\cvhonor
	{HashiCorp Ambassador} % Award
	{HashiCorp} % Event
	{} % Location
	{2022} % Date(s)

	%---------------------------------------------------------
\end{cvhonors}

%-------------------------------------------------------------------------------
%	SECTION TITLE
%-------------------------------------------------------------------------------
\cvsection{Skills}


%-------------------------------------------------------------------------------
%	CONTENT
%-------------------------------------------------------------------------------
\begin{cvskills}

	\cvskill
	{Programming} % Category
	{Python (Numpy, Scipy, Pandas, Pola-rs, Matplotlib, Seaborn, scikit-learn, PyTorch, JAX), Julia (DataFrames, JuMP, Plots), C++} % Skills

	\cvskill
	{Software}
	{CLI/Unix, Linux (Debian), Virtualization (Proxmox, LXC), Docker, ZFS, S3 Storage, Git, Wireguard VPN, Vim/Neovim, \LaTeX, Pandoc Markdown}

	\cvskill
	{Data}
	{OLS, ARMA, Machine Learning, Deep Learning, Fourier Analysis, Maximum Likelihood, Generalized Method Moments}
	%---------------------------------------------------------
	\cvskill
	{Languages} % Category
	{French (Native), English (Fluent), Spanish (Proficient)} % Skills

	\cvskill
	{Hobbies}
	{Coffee Barista, Cooking, Soccer, Travelling, Self-Hosting, Reading about History, Physics Videos}
	%---------------------------------------------------------
\end{cvskills}

% %-------------------------------------------------------------------------------
%	SECTION TITLE
%-------------------------------------------------------------------------------
\cvsection{Presentation}


%-------------------------------------------------------------------------------
%	CONTENT
%-------------------------------------------------------------------------------
\begin{cventries}

%---------------------------------------------------------
  \cventry
    {Presenter for <Hosting Web Application for Free utilizing GitHub, Netlify and CloudFlare>} % Role
    {DevFest Seoul by Google Developer Group Korea} % Event
    {Seoul, S.Korea} % Location
    {Nov. 2017} % Date(s)
    {
      \begin{cvitems} % Description(s)
        \item {Introduced the history of web technology and the JAM stack which is for the modern web application development.}
        \item {Introduced how to freely host the web application with high performance utilizing global CDN services.}
      \end{cvitems}
    }

%---------------------------------------------------------
  \cventry
    {Presenter for <DEFCON 20th : The way to go to Las Vegas>} % Role
    {6th CodeEngn (Reverse Engineering Conference)} % Event
    {Seoul, S.Korea} % Location
    {Jul. 2012} % Date(s)
    {
      \begin{cvitems} % Description(s)
        \item {Introduced CTF(Capture the Flag) hacking competition and advanced techniques and strategy for CTF}
      \end{cvitems}
    }

%---------------------------------------------------------
  \cventry
    {Presenter for <Metasploit 101>} % Role
    {6th Hacking Camp - S.Korea} % Event
    {S.Korea} % Location
    {Sep. 2012} % Date(s)
    {
      \begin{cvitems} % Description(s)
        \item {Introduced basic procedure for penetration testing and how to use Metasploit}
      \end{cvitems}
    }

%---------------------------------------------------------
\end{cventries}

% %-------------------------------------------------------------------------------
%	SECTION TITLE
%-------------------------------------------------------------------------------
\cvsection{Writing}


%-------------------------------------------------------------------------------
%	CONTENT
%-------------------------------------------------------------------------------
\begin{cventries}

%---------------------------------------------------------
  \cventry
    {Founder \& Writer} % Role
    {A Guide for Developers in Start-up} % Title
    {Facebook Page} % Location
    {Jan. 2015 - PRESENT} % Date(s)
    {
      \begin{cvitems} % Description(s)
        \item {Drafted daily news for developers in Korea about IT technologies, issues about start-up.}
      \end{cvitems}
    }

%---------------------------------------------------------
  \cventry
    {Undergraduate Student Reporter} % Role
    {AhnLab} % Title
    {S.Korea} % Location
    {Oct. 2012 - Jul. 2013} % Date(s)
    {
      \begin{cvitems} % Description(s)
        \item {Drafted reports about IT trends and Security issues on AhnLab Company magazine.}
      \end{cvitems}
    }

%---------------------------------------------------------
\end{cventries}

% %-------------------------------------------------------------------------------
%	SECTION TITLE
%-------------------------------------------------------------------------------
\cvsection{Program Committees}


%-------------------------------------------------------------------------------
%	CONTENT
%-------------------------------------------------------------------------------
\begin{cvhonors}

%---------------------------------------------------------
  \cvhonor
    {Problem Writer} % Position
    {2016 CODEGATE Hacking Competition World Final} % Committee
    {S.Korea} % Location
    {2016} % Date(s)

%---------------------------------------------------------
  \cvhonor
    {Organizer \& Co-director} % Position
    {1st POSTECH Hackathon} % Committee
    {S.Korea} % Location
    {2013} % Date(s)

%---------------------------------------------------------
  \cvhonor
    {Staff} % Position
    {7th Hacking Camp} % Committee
    {S.Korea} % Location
    {2012} % Date(s)

%---------------------------------------------------------
  \cvhonor
    {Problem Writer} % Position
    {1st Hoseo University Teenager Hacking Competition} % Committee
    {S.Korea} % Location
    {2012} % Date(s)

%---------------------------------------------------------
  \cvhonor
    {Staff \& Problem Writer} % Position
    {JFF(Just for Fun) Hacking Competition} % Committee
    {S.Korea} % Location
    {2012} % Date(s)

%---------------------------------------------------------
\end{cvhonors}

% %-------------------------------------------------------------------------------
%	SECTION TITLE
%-------------------------------------------------------------------------------
\cvsection{Extracurricular Activity}


%-------------------------------------------------------------------------------
%	CONTENT
%-------------------------------------------------------------------------------
\begin{cventries}

%---------------------------------------------------------
  \cventry
    {Core Member} % Affiliation/role
    {B10S (B1t 0n the Security, Underground hacker team)} % Organization/group
    {S.Korea} % Location
    {Nov. 2011 - PRESENT} % Date(s)
    {
      \begin{cvitems} % Description(s) of experience/contributions/knowledge
        \item {Gained expertise in penetration testing areas, especially targeted on web application and software.}
        \item {Participated on a lot of hacking competition and won a good award.}
        \item {Held several hacking competitions non-profit, just for fun.}
      \end{cvitems}
    }

%---------------------------------------------------------
  \cventry
    {Member} % Affiliation/role
    {WiseGuys (Hacking \& Security research group)} % Organization/group
    {S.Korea} % Location
    {Jun. 2012 - PRESENT} % Date(s)
    {
      \begin{cvitems} % Description(s) of experience/contributions/knowledge
        \item {Gained expertise in hardware hacking areas from penetration testing on several devices including wireless router, smartphone, CCTV and set-top box.}
        \item {Trained wannabe hacker about hacking technique from basic to advanced and ethics for white hackers by hosting annual Hacking Camp.}
      \end{cvitems}
    }

%---------------------------------------------------------
  \cventry
    {Core Member \& President at 2013} % Affiliation/role
    {PoApper (Developers' Network of POSTECH)} % Organization/group
    {Pohang, S.Korea} % Location
    {Jun. 2010 - Jun. 2017} % Date(s)
    {
      \begin{cvitems} % Description(s) of experience/contributions/knowledge
        \item {Reformed the society focusing on software engineering and building network on and off campus.}
        \item {Proposed various marketing and network activities to raise awareness.}
      \end{cvitems}
    }

%---------------------------------------------------------
  \cventry
    {Member} % Affiliation/role
    {PLUS (Laboratory for UNIX Security in POSTECH)} % Organization/group
    {Pohang, S.Korea} % Location
    {Sep. 2010 - Oct. 2011} % Date(s)
    {
      \begin{cvitems} % Description(s) of experience/contributions/knowledge
        \item {Gained expertise in hacking \& security areas, especially about internal of operating system based on UNIX and several exploit techniques.}
        \item {Participated on several hacking competition and won a good award.}
        \item {Conducted periodic security checks on overall IT system as a member of POSTECH CERT.}
        \item {Conducted penetration testing commissioned by national agency and corporation.}
      \end{cvitems}
    }

%---------------------------------------------------------
  \cventry
    {Member} % Affiliation/role
    {MSSA (Management Strategy Club of POSTECH)} % Organization/group
    {Pohang, S.Korea} % Location
    {Sep. 2013 - Jun. 2017} % Date(s)
    {
      \begin{cvitems} % Description(s) of experience/contributions/knowledge
        \item {Gained knowledge about several business field like Management, Strategy, Financial and marketing from group study.}
        \item {Gained expertise in business strategy areas and inisght for various industry from weekly industry analysis session.}
      \end{cvitems}
    }

%---------------------------------------------------------
\end{cventries}



%-------------------------------------------------------------------------------
\end{document}
