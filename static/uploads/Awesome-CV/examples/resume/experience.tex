%-------------------------------------------------------------------------------
%	SECTION TITLE
%-------------------------------------------------------------------------------
\cvsection{Research Experience}


%-------------------------------------------------------------------------------
%	CONTENT
%-------------------------------------------------------------------------------
\begin{cventries}
	\cventry
	{joint with Anmol Bhandari, David Evans, Mikhail Golosov and Thomas J. Sargent} % Degree
	{\href{https://static1.squarespace.com/static/54c19f18e4b0ef5f4b9f8dae/t/6328ee5213a65c43a48423f8/1663626851912/abegs4draft.pdf}{Managing Public Portfolios} \textit{(Revise \& Resubmit at The Journal of Political Economy)}}  % Institution
	{\href{https://www.nber.org/papers/w30501}{NBER Working Paper \#30501}} % Location
	{2022} % Date(s)
	{\begin{cvitems}
			\item I helped develop a quantitative framework for optimally managing public portfolios for various market structures for financial assets, and trading frictions.
			\item I contributed to characterize numerically the optimal US maturity structure using macro and bonds market data. This calibration shows that the \textit{interest rate risk} should shape the US debt portfolio.
			\item I implemented an affine dynamic asset pricing model of the US government bond market in Python (Pandas, Numpy, Scipy).
		\end{cvitems}}

	\cventry
	{Lars Peter Hansen \& Thomas J. Sargent, Ufuk Akcigit}
	{Research Assistant}
	{University of Chicago}
	{2019-2020}
	{\begin{cvitems}
			\item I helped develop a quantitative model of the optimal taxation for R\&D Policies in the US using Numpy and Scipy.
			\item I checked proofs on statistical uncertainty, model misspecification, and time inconsistency, using concepts of statistical divergences, information geometry, and entropy.
		\end{cvitems}
	}
\end{cventries}
